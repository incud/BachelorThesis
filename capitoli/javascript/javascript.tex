\chapter{Linguaggio Javascript}

Il linguaggio di programmazione e di scripting Javascript è un linguaggio \emph{multiparadigma} (ad oggetti, imperativo e funzionale), \emph{dinamico} ed \emph{interpretato}. Nasce nel 1995 per la creazione di software web lato client e ad oggi è il più diffuso linguaggio di programmazione al mondo.

Il suo essere interpretato, la tipizzazione debole e i casting impliciti portano a produrre software pieno di errori anche difficilmente localizzabili. Queste ragioni motivano l'esistenza del tool. 

\section{Tipi di dato}

I tipi di dato del linguaggio sono:
\begin{itemize}
    \item numeri: valori in virgola mobile a 64-bit;
    \item booleani: \texttt{true} e \texttt{false};
    \item stringhe: sequenze di caratteri Unicode racchiusi tra apici singoli o doppi;
    \item \texttt{null}: tipo il cui unico valore è \texttt{null};
    \item array: sequenza di elementi omogenei;
    \item oggetti: collezioni di variabili dette proprietà.
\end{itemize}

\subsection{Proprietà degli oggetti}

Gli oggetti sono formati da un insieme di proprietà alle quali si accede tramite \texttt{oggetto.proprieta} oppure \texttt{oggetto["proprieta"]}. Le proprietà possono essere aggiunte anche dinamicamente. Gli oggetti sono creati dai costruttori.

\subsection{Valore undefined}

Esiste un valore \emph{undefined}. Il test \texttt{x == undefined} è vero se \texttt{x} è stata dichiarata senza mai assegnarne un valore. Il test \texttt{x.prop == undefined} è vero se \texttt{x} è un oggetto che non ha definito la proprietà \texttt{prop}.

\section{Visibilità}

Le variabili possono essere dichiarate per mezzo di parole chiave diverse che ne definiscono la visibilità:
\begin{itemize}
    \item \texttt{let}: la variabile è interna al blocco;
    \item \texttt{const}: la variabile è interna al blocco ma non riassegnabile (posso però ancora aggiungere e modificare proprietà);
    \item \texttt{var}: la variabile è interna alla definizione di funzione corrente.
\end{itemize}

\section{Funzioni}

Le funzioni vengono definite tramite la sintassi \mintinline{javascript}{function fnname(args) {body}}. Possono essere ricorsive. Possono essere scritte nella notazione compatta \emph{arrow}, assegnate alle variabili e ritornate.
\begin{javascriptcode}
function hello(name) {
    console.log("hello " + name);
}
var f1 = hello;
var f2 = () => { console.log("hello arrow " + name); };
f1("bob"); f2("alice"); // hello bob\nhello arrow alice
\end{javascriptcode}

\subsection{Parametro \texttt{this}}

La notazione delle parentesi tonde per la chiamata a funzione è lo zucchero sintattico della primitiva \texttt{call}. E' equivalente chiamare \mintinline{javascript}{hello("bob")} ed \mintinline{javascript}{hello.call(undefined, "bob")}. Il primo parametro della primitiva è il parametro implicito \texttt{this}.
\begin{javascriptcode}
function hello(name) {
    console.log("hello " + name);
}
hello("bob"); // hello.call(undefined, "bob");

var obj = { };
obj.greet = hello;
obj.greet("bob"); // hello.call(obj, "bob");
\end{javascriptcode}
Il parametro implicito equivale all'oggetto chiamante, se la funzione è definita all'interno di un oggetto. Altrimenti è settato ad \texttt{undefined}.

\subsection{Creazione di oggetti}\label{sec:js:createobj}

La creazione di oggetti avviene per messo di costruttori, che sono funzioni precedute dalla parola chiave \texttt{new}. Queste generano un nuovo oggetto vuoto e lo popolano di proprietà attraverso il riferimento \texttt{this}.
\begin{javascriptcode}
function Rectangle(width, height) {
    this.w = width;
    this.h = height;
    this.toString = function() { 
        console.log("Rect size: " + this.w + " x " + this.h); 
    }
}
let rect = new Rectangle(3, 4);
rect.w = 15;
rect.toString(); // "Rect size: 15 x 4"
\end{javascriptcode}

\subsection{Modificatori di accesso}

Il linguaggio non ha la parola chiave \texttt{private} sulle proprietà degli oggetti. Le variabili private sono ottenute definendole tramite la parola chiave \texttt{var} e racchiudendole all'interno di closure.

\section{Prototype}

Nella Sezione~\ref{sec:js:createobj} siamo riusciti a creare un oggetto. La proprietà \texttt{toString} è una funzione che viene copiata e ridefinita per tutti gli oggetti \texttt{Rectangle}. Vorremmo ce ne fosse un'unica copia. Viene introdotto il prototype.
\begin{javascriptcode}
Rectangle.prototype.printInfo = function() { 
    console.log("Inside prototype size: " + this.w + " x " + this.h); 
}
let rect = new Rectangle(3, 4);
rect.printInfo(); // "Inside prototype size: 3 x 4"
\end{javascriptcode}
Quando un oggetto viene creato, questo eredita tutte le proprietà (variabili e funzioni) definite nell'oggetto \texttt{prototype} del suo costruttore. L'oggetto mantiene un puntatore al \texttt{prototype} nelle proprietà \mintinline{javascript}{__proto__}. Eventuali modifiche al \texttt{prototype} si riflettono su tutti gli oggetti creati con tale costruttore. Le proprietà locali dell'oggetto nascondono eventuali proprietà ereditate da prototype se hanno lo stesso nome. 






























