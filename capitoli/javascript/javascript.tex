\chapter{Linguaggio Javascript}

Il linguaggio di programmazione e di scripting Javascript è un linguaggio \emph{multiparadigma} (ad oggetti, imperativo e funzionale), \emph{dinamico} ed \emph{interpretato}. Nasce nel 1995 per la creazione di software web lato client e ad oggi è il più diffuso linguaggio di programmazione al mondo\footnote{\url{http://www.businessinsider.com/the-9-most-popular-programming-languages-according-to-the-facebook-for-programmers-2017-10?IR=T#1-javascript-15}}. 

Il suo essere interpretato, la tipizzazione debole e i casting impliciti portano a produrre software pieno di errori anche difficilmente localizzabili. Queste ragioni motivano l'esistenza del tool. 

\section{Tipi di dato}

I tipi base sono: \emph{numeri}, \emph{booleani} e \emph{stringhe}. I numeri sono tutti valori a 64-bit in virgola mobile. Le stringhe sono sequenze di caratteri Unicode racchiusi tra le doppie virgolette o le virgolette singole ed ammettono alcune sequenze di escape. I caratteri sono stringhe di lunghezza 1. I valori booleani sono \texttt{true} e \texttt{false}. 

I tipi di dato composti sono: \emph{oggetti} e \emph{matrici}. Gli oggetti sono formati da un insieme di proprietà alle quali si accede tramite \texttt{oggetto.proprieta} oppure \texttt{oggetto["proprieta"]}. Le proprietà possono essere aggiunte anche dinamicamente. Gli oggetti sono creati dai costruttori.

Esiste un tipo di dato \emph{Null} il cui unico valore è \texttt{null}. 

Esiste un valore \emph{undefined}. Il test \texttt{x == undefined} è vero se \texttt{x} è stata dichiarata senza mai assegnarne un valore. Il test \texttt{x.prop == undefined} è vero se \texttt{x} è un oggetto che non ha definito la proprietà \texttt{prop}.

\section{Istruzioni}

\section{Costruttori}

\section{Prototype}



