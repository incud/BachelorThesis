\newcommand{\primrec}{\mathcal{P\!\!R}}

\chapter{Analisi statica di programmi}

L'analisi di programmi nasce per verificare la correttezza del software sviluppato e la sua conformità alle specifiche dello stesso. Le due principali tecniche adoperate sono l'\emph{analisi dinamica} e l'\emph{analisi statica}.

L'analisi dinamica consiste nel fissare alcuni input $i_1, ..., i_n$ ed i corrispondenti valori di output previsti $o_1, ..., o_n$ e verificare che l'output nel nostro programma $P$ sia tale che $\forall k \in [1,n] \;\; P(i_k) = o_k$. Il vantaggio principale di questa tecnica è la sua facilità di utilizzo: per testare la correttezza del programma fissato un input non occorre software aggiuntivo, basta eseguire il programma. Lo svantaggio principale è il non avere alcuna garanzia di correttezza per qualsiasi altro input che non sia tra quelli testati. Un esempio di analisi dinamica sono gli \emph{unit test}. 

L'analisi statica, a differenza della tecnica precedente, non esegue il codice. Si avvale di un software detto \emph{analizzatore} che prende in ingresso il \emph{codice sorgente del software} ed in output fornisce delle \emph{invarianti} o \emph{asserzioni} o \emph{proprietà} che valgono per ogni possibile input. Il vantaggio principale è la garanzia di correttezza che l'analizzatore fornisce quando riesce ad inferire la o le proprietà che devono valere per il codice. 

Gli svantaggi principali sono due: in primo luogo, occorre utilizzare software aggiuntivo quale l'analizzatore. In secondo luogo, un importante teorema limita fortemente i risultati che possiamo ottenere con questa tecnica: il \emph{teorema di Rice} dice infatti che non è possibile decidere se una proprietà \emph{non banale} vale o meno per un dato programma. 

\section{Macchine di Turing e funzioni parziali ricorsive}

\begin{definition}[Programma]
Un \emph{programma} è una sequenza finita di istruzioni codificate all'interno di un certo formalismo.
\end{definition}

Tra i formalismi tramite i quali codifico i programmi, due dei più importanti sono le \emph{Macchine di Turing} (MdT) e le \emph{funzioni parziali ricorsive}. 

Il primo è una macchina formata da un \emph{automa a stati finiti} ed un \emph{nastro infinito di simboli} con una \emph{testina} che punta alla prossima cella da leggere. Questa macchina esegue una certa azione $\delta(q,x)$ che dipende dal simbolo $x$ letto dalla testina e dallo stato corrente $q$ dell'automa. Le azioni possibili sono solamente scrivere un simbolo nella cella puntata dalla testina, muovere quest'ultima a destra o a sinistra, e cambiare stato.

\begin{definition}[Macchina di Turing (da {\cite[p.~338]{hopcroft09}})]
Descriviamo una TM [\emph{Turing Machine} ndr] come la settupla \(\langle Q, \Sigma, \Gamma, \delta, q_0, B, F \rangle\) con 
\begin{itemize}
    \item $Q$ insieme finito degli stati di controllo.
    \item $\Sigma$ insieme finito dei simboli in input.
    \item $\Gamma$ insieme completo dei simboli di nastro, $\Sigma$ è sempre sottoinsieme di $\Gamma$;
    \item $\delta$ funzione di transizione. Gli argomenti di $\delta(q,x)$ sono uno stato $q$ ed un simbolo di nastro $x$. Il valore di  $\delta(q,x)$, se definito, è una tripla $(p, y, d)$, dove 
    \begin{itemize}
        \item $p \in Q$ è lo stato successivo;
        \item $y$ è il simbolo di $\Gamma$ scritto nella cella guardata, al posto di qualunque simbolo vi fosse;
        \item $d$ è la direzione, $L$ o $R$, e segnala la direzione in cui si muove la testina.
    \end{itemize}
    \item $q_0$ stato iniziale del controllo, è un elemento di Q.
    \item $B$ simbolo \emph{blank}, si trova in $\Gamma$ ma non in $\Sigma$, cioè non è un simbolo di input. Inizialmente compare in tutte le celle del nastro tranne quelle (finite) che contengono i simboli in input.
    \item $F$ insieme degli stati finali o accentati, un sottoinsieme di $Q$.
\end{itemize}    
\end{definition}

La macchina di Turing incapsula il \emph{concetto intuitivo di programma}: un programma è tale se può essere codificato all'interno di una MdT. Un formalismo equivalente a questo è chiamato \emph{classe delle funzioni parziali ricorsive}.

\begin{definition}[Funzioni primitive ricorsive (da {\cite[p.~9]{soare87}})]
La classe $\mathcal{C}$ delle funzioni primitive ricorsive è il più piccolo insieme chiuso rispetto al seguente schema.
\begin{itemize}
    \item La funzione successore $\lambda x[x + 1]$ è in $\mathcal{C}$.
    \item Le funzioni costante $\lambda x_1 ... x_n[m]$ è in $\mathcal{C}$.
    \item Le funzioni identità (dette anche proiezioni) $\lambda x_1, ... x_n [x_i]$ sono in $\mathcal{C}$ per $1 \le i \le n$.
    \item (Composizione) Se $g_1, ..., g_m, n \in \mathcal{C}$ allora 
    \[ f(x_1, ..., x_n) = h(g_1(x_1, ..., x_n), ..., g_m(x_1, ..., x_n)) \in \mathcal{C}\]
    con $g_1, ... g_m$ funzioni di $n$ variabili ed $h$ funzione di $m$ variabili.
    \item (Ricorsione primitiva) Se $g, h \in \mathcal{C}$ e $n \ge 1$ allora $f \in \mathcal{C}$ con
    \begin{IEEEeqnarray*}{rCl}
        f(0, x_2, ..., x_n)       & = & g(x_2, ..., x_n) \\
        f(x_1 + 1, x_2, ..., x_n) & = & h(x_1, x_2, ..., x_n, f(x_1, ..., x_n))
    \end{IEEEeqnarray*}
    assumendo $g, h$ funzioni di $n-1$ ed $n+1$ variabili rispettivamente.
\end{itemize}
\end{definition}

\begin{definition}[Funzioni parziali ricorsive (da {\cite[p.~10]{soare87}})]
La classe $\primrec$ delle funzioni parziali ricorsive è il più piccolo insieme chiuso rispetto allo schema delle funzioni primitive ricorsive, e chiuso rispetto all'operazione di \emph{unbounded search}: se $\varphi(x_1, ..., x_n, y) \in \primrec$ e
\begin{IEEEeqnarray*}{rCll}
    \psi(x_1, ..., x_n) & = & \min y \; \{ & \varphi(x_1, ..., x_n, y) \downarrow = 0 \\
                        &   &  & \land \; \forall z \le y : \varphi(x_1, ..., x_n, z) \downarrow \}
\end{IEEEeqnarray*}
allora $\psi \in \primrec$. [Nota: $\varphi(x) \downarrow$ vuol dire che $\varphi$ termina con input $x$].
\end{definition}

I formalismi descritti sono equivalenti: a partire da una MdT posso derivare la funzione parziale ricorsiva che implementa, e viceversa. Una dimostrazione di questa proposizione si trova in [\cite[p.~15]{soare87}].

\section{Proprietà dei programmi}

E' possibile assegnare ad ogni funzione parziale ricorsiva $\phi$ un numero naturale $i$ detto \emph{indice}. Allo stesso modo, è possibile assegnare un indice ad ogni macchina di Turing. 

\begin{theorem}[Teorema della enumerazione {(da \cite[pp.~15-16]{soare87})}]
Esiste una funzione parziale ricorsiva di due variabili $\psi(e,x)$ che enumera l'insieme $\{ \varphi_n \}_{n \in \N}$ delle funzioni parziali ricorsive, cioè $\psi(e,x)$ è tale che $\psi(e,x) = \varphi_e(x)$.
\end{theorem}

\begin{definition}[Proprietà {(da \cite[pp.~173]{giaco17})}]
Una proprietà $\Pi$ sulle MdT (o sulle funzioni parziali ricorsive) è un qualsiasi sottoinsieme di $\{ M_n \}_{n \in \N}$. Con abuso di notazione, intenderemo con proprietà qualsiasi sottoinsieme di N, intendendo l’insieme costituito da indici di MdT (o di funzioni parziali ricorsive).
\end{definition}

Alcuni esempi di proprietà sono $\Pi_1 = \{ i \mid M_i \text{ non effettua divisioni per zero} \}$ oppure $\Pi_2 = \{ i \mid \text{il programma di } M_i \text{ contiene un \texttt{while}} \}$.

\section{Limiti dell'analisi statica}

\begin{definition}[Insieme ricorsivo o decidibile {(da \cite[pp.~164]{giaco17})}]
Un insieme (o proprietà) $\Pi$ è detto ricorsivo oppure decidibile se esiste una funzione ricorsiva \emph{totale} $p$ tale che $p(x) = 1$ se $x \in \Pi$ e $p(x) = 0$ se $x \not\in \Pi$.
\end{definition}

Le proprietà sono estensionali se riguardano in qualche modo il comportamento del programma, quindi le proprietà semantiche. Un esempio è: $\Pi = \{ M_i \mid M_i \text{ termina sempre} \}$. Le proprietà intensionali riguardano \emph{com'è fatto} il programma, quindi le proprietà sintattiche. Un esempio è $\Pi = \{ M_i \mid \text{il programma di } M_i \text{ contiene }$ $\text{un } \texttt{while} \}$.

\begin{definition}[Proprietà estensionale {(da \cite[pp.~175]{giaco17})}]
Sia $\Pi$ una proprietà sulle MdT. $\Pi$ si dice estensionale (o chiusa per uguaglianza estensionale) se per ogni $x,y\in\N$ si ha
\[ \left( x \in \Pi \; \land \; \varphi_x = \varphi_y \right) \to y \in \Pi \]
\end{definition}

[Nota: La notazione $\varphi_x = \varphi_y$ abbrevia $\forall z . \varphi_x(z)= \varphi_y(z)$.] La maggior parte proprietà semantiche (estensionali) che vorremmo provare non sono decidibili.

\begin{theorem}[Corollario del Teorema di Rice (da {\cite[p.~7]{rice53}})]
Ogni proprietà estensionale è decidibile se e solo se è banale (uguale all'insieme vuoto oppure a tutto $\N$).
\end{theorem}

\section{Soundness e completeness}

Una volta sviluppato il nostro \emph{analizzatore statico}, possiamo testarne due proprietà: la \emph{soundness} e la \emph{completeness}. 

L'analizzatore gode della proprietà di \emph{soundness} o \emph{consistenza} se approssima per eccesso il comportamento del programma. In questo caso, dati in input $\Pi$ ed $M$, riporto tutte le violazioni della proprietà $\Pi$ ma alcune di queste potrebbero essere falsi allarmi.

L'analizzatore gode della proprietà di \emph{completeness} o \emph{completezza} se approssima per difetto il comportamento del programma.
In questo caso, dati in input $\Pi$ ed $M$, tutte le violazioni della proprietà $\Pi$ che riporto sono \emph{vere violazioni}, è però possibile avere falsi negativi. 

\begin{table}[htbp]
    \centering
    \begin{tabular}{c p{2.1cm} p{2.1cm} p{2.1cm} }
    \toprule
    & \multicolumn{3}{c}{{Proprietà dell'analizzatore}} \\[0.5em]\cmidrule{2-4}
    Output & Soundness & Completeness & Soundness e completeness \\\midrule
    $\Pi$ vale     & Sicuramente $\Pi$ vale & Non posso dire nulla & Sicuramente $\Pi$ vale \\
    $\Pi$ non vale & Non posso dire nulla   & Sicuramente $\Pi$ non vale & Sicuramente $\Pi$ non vale \\
    \bottomrule
    \end{tabular}
    \caption{Tabella riassuntiva di soundness e completeness}
\end{table}

Per il teorema di Rice, non è possibile avere entrambe le proprietà. Preferiamo avere un analizzatore solo consistente piuttosto che solo completo in quanto fornisce garanzie sulla correttezza.