% math-generic
\usepackage{mathtools,amsmath,amssymb,amsfonts,galois}
\usepackage{IEEEtrantools}
\usepackage{textcomp}
\usepackage{stmaryrd}

% math-theorems
\usepackage{amsthm}
\newtheorem{theorem}{Teorema}
\theoremstyle{definition}
\newtheorem{definition}[theorem]{Definizione}
\newtheorem{example}[theorem]{Esempio}

\newcommand{\N}{\mathbb{N}}
\newcommand{\Z}{\mathbb{Z}}

\newcommand{\semantics}[1]{\llbracket #1 \rrbracket}
\newcommand{\set}[1]{\mathrm{#1}}
\newcommand{\term}[1]{\mathrm{#1}}
\newcommand{\fun}[1]{\mathit{#1}}
\newcommand{\struct}[1]{\langle #1 \rangle}

\usepackage{xparse}

\NewDocumentCommand{\NewCustomOperator}{o o m m m}{
    \IfNoValueTF{#1}{
        #3
    }{
        \IfNoValueTF{#2}{
            #4 #1
        }{
            #1 #5 #2
        }
    }
}

\NewDocumentCommand{\meet}{o o}{
    \NewCustomOperator[#1][#2]{\sqcap}{\bigsqcap}{\sqcap}
}

\NewDocumentCommand{\join}{o o}{
    \NewCustomOperator[#1][#2]{\sqcup}{\bigsqcup}{\sqcup}
}

\NewDocumentCommand{\wide}{o o}{
    \NewCustomOperator[#1][#2]{\nabla}{\nabla}{\nabla}
}